\documentclass[a4paper, 12pt]{article}

\usepackage[czech]{babel}
\usepackage[T1]{fontenc}
\usepackage[a4paper, margin=2cm]{geometry}
\usepackage{amsmath}
\usepackage{amssymb}
% \usepackage{amsthm}
% package - mathtool

\usepackage{graphicx}
\usepackage[unicode]{hyperref}


\newcommand*\diff{\mathop{}\!\mathrm{d}}

\title{Matematika}
\author{Petr Kotlan}
\date{}

\DeclareMathOperator{\tg}{tg}

\begin{document}

\maketitle

% \tableofcontents

Gon. funkce:

$\sin{x}$, $\cos{x}$, $\tg{x}$, $\tan{x}$, $\cot{x}$

$$ \cos{x} $$

\begin{equation*}
    \cos{x} = 0
\end{equation*}


\begin{equation}
    \sin{x}
\end{equation}

\begin{equation}
    2x + 3 = 10
\end{equation}

Zlomek:

\begin{equation}
    \frac{x + 1}{x + 2}
\end{equation}

Tady v tom řádku vysázím zlomek: $\frac{x + 1}{x + 2}$.

Mocnina:
$$ \left(\frac{a}{b}\right)^2 $$

Odmocnina:
$$ \sqrt{2}, \sqrt[3]{2} $$

$$ \left(
    \frac{\sqrt{2}}{\sqrt[3]{2}}
    \right)^{k + 1} $$

$$ \frac{1}{1 + \frac{x - 1}{6}} $$

Derivace:
$$ y'=2x $$

Parciální derivace:
$$ \frac{\partial}{\partial x} f(a)$$
$$ \frac{\partial^2}{\partial x^2} f(x)$$

Integrály:
$$ \int \ln{x} \diff{x} $$

$$ \int_{b}^{a}\limits x^3 + 2x \diff{x} $$  % \limits udělá správně indexy

$$ \iint x^3 + 2x \diff{x} $$

$$ \int \dots \int^{(6)} 2x \diff{x} $$

% \[ \sin{x} \]
% \( \cos{x} \)

Suma:
$$ \sum_{i=1}^{n}  \frac{1}{2^n} $$

Abeceda:
$$ \alpha \beta \gamma \delta \epsilon \zeta \eta \theta \iota \kappa \lambda \mu \nu \xi \pi \rho \sigma \tau \upsilon \varphi \chi \psi \omega  $$

Limita:
$$ \lim_{n \to \infty} \frac{1}{2^n} $$

Kombinační číslo:
$$ \binom{n}{k} $$

Kvantifikátory:
$$ \forall{x} \in \mathbb{R},\ \exists{y} \in \mathbb{R}; \ y = x ^ 3$$

Desetinná čísla:
$$ 3{,}125 $$

Indexy:
$$ x_1, x_2, \dots, x_n, x_{n+1} $$

Tečky:
$$ \dots, \cdots, \ldots $$

Matice:
$$
    \left(
    \begin{matrix}
        1 & 2 & 3 \\
        a & b & c \\
        7 & 8 & 9
    \end{matrix}
    \right]
$$

$$
    \begin{pmatrix}
        1 & 2 & 3 \\
        a & b & c \\
        7 & 8 & 9
    \end{pmatrix}
$$

$$
    \begin{bmatrix}
        1 & 2 & 3 \\
        a & b & c \\
        7 & 8 & 9
    \end{bmatrix}
$$

$$
    \begin{Bmatrix}
        1 & 2 & 3 \\
        a & b & c \\
        7 & 8 & 9
    \end{Bmatrix}
$$

$$
    \begin{vmatrix}
        1 & 2 & 3 \\
        a & b & c \\
        7 & 8 & 9
    \end{vmatrix}
    \quad % nemůžem dělat mezery v matematické prostředí
    \begin{Vmatrix}
        1 & 2 & 3 \\
        a & b & c \\
        7 & 8 & 9
    \end{Vmatrix}
$$

$$
    \left \lceil
    \begin{matrix}
        1 & 2 & 3 \\
        a & b & c \\
        7 & 8 & 9
    \end{matrix}
    \right \rceil
$$

$$
    \left \lfloor
    \begin{matrix}
        1 & 2 & 3 \\
        a & b & c \\
        7 & 8 & 9
    \end{matrix}
    \right \rfloor
$$

$$
    \left \langle
    \begin{matrix}
        1 & 2 & 3 \\
        a & b & c \\
        7 & 8 & 9
    \end{matrix}
    \right \rangle
$$

$$
    \left(
    \begin{array}{cc|c} % blbě závorky
            1 & 2 & 3 \\
            a & b & c \\
            \hline
            7 & 8 & 9
        \end{array}
    \right)
$$

$$
    \left(
    \begin{matrix}
            1 & 2 & 3 \\
            a & b & c \\
            7 & 8 & 9
        \end{matrix}
    \right)
$$

$$
    \left(
    \begin{smallmatrix}
            1 & 2 & 3\\
            a & b & c\\
            7 & 8 & 9
        \end{smallmatrix}
    \right)
$$

Zalomená rovnice:

\begin{equation}
    \begin{split}
        1 &=  \cos ^ 2 x + \sin ^2 x \\
        & =  \tg ^2 x + 17
    \end{split}
\end{equation}

\begin{multline}
    1 =  \cos ^ 2 x + \sin ^2 x \\
    =  \tg ^2 x + 17
\end{multline}

Soustavy rovnic:

\begin{align} \label{TriSoust}
    2x + 3y & = -5  & 3x + 5y   & = -121  \\
    x - 7y  & = 121 & 17x + 19y & = 0{,}5
\end{align}

Label:

\begin{equation} \label{EulRce}
    e^{\mathrm{i} \pi} + 1 = 0
\end{equation}

Za velmi krásnou rovnici je považována rovnice \ref{EulRce}, ketrá se nachází na straně \pageref{EulRce}.
Co se stane, když se odvoláme na něco co má jeden štítek, ale dvě čísla, jako je tomu u \ref{TriSoust}?
Jak to udělat, aby se u soustavy zobrazovalo jen jedno číslo?
Jak se odkázat na konkrétní řádek?

Výroková logika:
$p \wedge q \vee r \Rightarrow p \Leftrightarrow q \neg r$
$ \emptyset A \subset B \subseteq B 2 \leq 3 \geq 1$


\end{document}
