\documentclass[a4paper, 12pt]{article}


\usepackage[czech]{babel}
\usepackage[T1]{fontenc}
\usepackage{graphicx}
\usepackage[unicode]{hyperref} % unicode propisuje diakritiku do menu
\usepackage{listings}
\usepackage{xcolor}
\usepackage{pagecolor}

\usepackage{lipsum}
\usepackage{marginnote}
\usepackage[a4paper, marginparwidth=3cm, marginparsep=5mm,left=3cm, right=4cm, top=2cm, bottom=4cm]{geometry}
\usepackage{fancyhdr}

\definecolor{codegreen}{rgb}{0,0.6,0}
\definecolor{codegray}{rgb}{0.5,0.5,0.5}
\definecolor{codepurple}{rgb}{0.58,0,0.82}
\definecolor{backcolour}{rgb}{0.95,0.95,0.95}
\definecolor{ballblue}{rgb}{0.13, 0.67, 0.8}
\definecolor{moleskin}{HTML}{FFF8DC}



\lstdefinestyle{mystyle}{
    backgroundcolor=\color{white},   
    commentstyle=\color{codegreen},
    keywordstyle=\color{ballblue},
    numberstyle=\tiny\color{codegray},
    stringstyle=\color{codepurple},
    basicstyle=\ttfamily\footnotesize,
    breakatwhitespace=false,         
    breaklines=true,                 
    captionpos=b,                    
    keepspaces=true,                 
    numbers=left,                    
    numbersep=5pt,                  
    showspaces=false,                
    showstringspaces=false,
    showtabs=false,                  
    tabsize=2
}

\lstset{style=mystyle}

\pagestyle{fancy}

\author{Petr Kotlan}
\title{Kód a pozicování textu}

\begin{document}

\pagecolor{moleskin}
\maketitle
\tableofcontents
\pagebreak

\section{Code}
\noindent Nějaký text předtím.
\begin{verbatim*}
    for i in range(10):
    print(i)
\end{verbatim*}
Nějaký text po tom.

\subsection{Code itself}
\begin{lstlisting}[language=Python, caption=Python example]
import numpy as np
    
def incmatrix(genl1,genl2):
    m = len(genl1)
    n = len(genl2)
    M = None #to become the incidence matrix
    VT = np.zeros((n*m,1), int)  #dummy variable
    
    #compute the bitwise xor matrix
    M1 = bitxormatrix(genl1)
    M2 = np.triu(bitxormatrix(genl2),1) 

    for i in range(m-1):
        for j in range(i+1, m):
            [r,c] = np.where(M2 == M1[i,j])
            for k in range(len(r)):
                VT[(i)*n + r[k]] = 1;
                VT[(i)*n + c[k]] = 1;
                VT[(j)*n + r[k]] = 1;
                VT[(j)*n + c[k]] = 1;
                
                if M is None:
                    M = np.copy(VT)
                else:
                    M = np.concatenate((M, VT), 1)
                
                VT = np.zeros((n*m,1), int)
    
    return M
    \end{lstlisting}

\marginnote{\color{red} Poznamka na okraj }
\pagebreak

\section{Poem}
\begin{verse}
    Ptal se jednou jeden pán.\\
    Kudy tudy do hajan?\\
    To se dáte podél plotu.\\
    od klímadel na dřímotu.

    V parku stojí socha,\\
    děvčete a hocha.\\
    S neskrývanou rozkoší\\
    každou noc tu sousoší.



\end{verse}

\end{document}